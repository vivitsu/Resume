%---------------------------------------------
%			RESUME
%---------------------------------------------
\documentclass[10pt]{article}

%---------------------------------------------
%	BASIC PACKAGE DECLARATIONS
%---------------------------------------------
\usepackage[top=0.3in, bottom=0.3in, left=0.3in, right=0.5in]{geometry}
%\usepackage{array, xcolor}
%\usepackage{mathpazo}
%\usepackage[parfill]{parskip} % Remove paragraph indentation
\usepackage{parskip}
\usepackage{hyperref}

\begin{document}
%---------------------------------------------
%		     TITLE
%---------------------------------------------
\begin{center}
\textbf{\Large Vivitsu Maharaja}
\date{}
\thispagestyle{empty}
\smallskip \\
%---------------------------------------------
%	     CONTACT INFORMATION
%---------------------------------------------
3800 SW 34\textsuperscript{th} Street, Apartment Z251, Gainesville, FL $\diamond$ (352) 278-5449 $\diamond$ vmaharaja@ufl.edu
\end{center}

%---------------------------------------------
%	           OBJECTIVE
%---------------------------------------------
\textbf{Objective}
\smallskip
\hrule
To obtain a position where I can leverage and enhance my skills \& experience while working on challenging problems in computer engineering \& software development.

%---------------------------------------------
%	           EDUCATION
%---------------------------------------------
\textbf{Education}
\smallskip
\hrule
{\bf Master of Science}, Electrical \& Computer Engineering \hfill May 2014\\
University of Florida, Gainesville, FL. {\bf GPA:} 3.16/4.0 \\
{\bf Bachelor of Engineering}, Electronics \& Communication \hfill May 2011\\
Dharmsinh Desai University, Nadiad, India. {\bf GPA:} 62/100

Coursework: Computer Architecture, Parallel Computer Architecture, Distributed Computing, Autonomic Computing, Cloud Computing, Computer Networks, Communication Systems.

%---------------------------------------------
%	           SKILLS
%---------------------------------------------
\textbf{Skills}
\smallskip
\hrule
Knowledge of Data Structures and Analysis of Algorithms\\
Programming Languages: x86 Assembly, C, Java, Go, MPI, CUDA \& \LaTeX\ \\
Technologies: Hadoop, Riak, Redis, Git, Subversion
%Operating Systems: GNU/Linux, OS X, Windows

%---------------------------------------------
%		WORK EXPERIENCE
%---------------------------------------------
\textbf{Experience}
\smallskip
\hrule
{\bf Advanced Computing \& Information Systems Lab}, University of Florida \hfill May 2013 - Present\\
\textit{Research Volunteer}
\begin{itemize}
%---------------------
%	PROJECT - 1
%---------------------
    \item Developed a Java application that fetches JSON objects from a URL and adds them to Riak. Bucket creation, key management \& interfacing with Riak are all managed by the application. %\smallskip \\
    \item Evaluated an Apache Solr system by benchmarking various parameters like indexing time, compression ratio, recall and the performance of the indexing algorithm.
%    \item Developing \& evaluating text-based information retrieval systems using frameworks like Hadoop, Riak \& Apache Solr.% \smallskip \\
\end{itemize}
{\bf Volansys Technologies}, Ahmedabad, India\hfill November 2011 - July 2012\\
\textit{Embedded Engineer}
\begin{itemize}
%---------------------
%	PROJECT - 1
%---------------------
    \item Developed a USB 2.0 (Enhanced Host Controller Interface) Host Controller driver in x86 assembly, as part of an application which allowed clients to PXE (Pre-boot eXecution Environment) boot via a network using an USB to Ethenet adapter. 
    \item Enhanced the driver to manage the complete state machine of the controller including device detection, power management and data transfer. %\smallskip \\
%    \item Developed software components to manage the PCI bus, negotiate ownership with the BIOS subsystem and added support for multiple vendor adapters, as part of the overall PXE application. %\smallskip \\
%Handled responsibility for all aspects of the project, including development, maintenance, documentation \& client interaction.\smallskip \\
%---------------------
%	PROJECT - 2
%---------------------
%    \item Developed a software feature for that would allow multiple broadcast domains in a wireless router to form a VLAN (Virtual LAN). The wireless router was based on the MIPS \& VxWorks platforms. (Team: 3 members)
\end{itemize}
%---------------------------------------------
%	          PROJECTS
%---------------------------------------------
\textbf{Projects}
\smallskip
\hrule
%---------------------
%	PROJECT - 1
%---------------------
\textbf{Distributed File System} \textit{using Java} \hfill August 2013 - December 2013
\begin{itemize}
    \item Designed and implemented a distributed, decentralized file system based on a peer-to-peer architecture
    \item Implemented consistency and fault-tolerance models to ensure high availability
    \item Designed and implemented a client module and an application that uses the file system to store data on a cluster
    \item Source code can be found at \url{https://github.com/vivitsu/Aether}
\end{itemize}
%---------------------
%	PROJECT - 2
%---------------------
\textbf{Web Service for Location Based Applications} \textit{using Go \& Redis} \hfill April 2013
\begin{itemize}
    \item Designed and implemented a secure web application that allows a user to view different resources on the web about his location, including Wikipedia links if the user is at a well-known point of interest.
    \item Created and managed a datastore that stored a user's account details, his location history \& a POI database.
    \item Project source code can be found at \url{https://bitbucket.org/vivitsu/goserve}.
\end{itemize}
%---------------------
%	PROJECT - 3
%---------------------
%{\bf Face Recognition using Artificial Neural Networks} \textit{using C \& CUDA}\hfill{\bf February 2013 - April 2013} \smallskip \\
%Designed and implemented an application that performed face recognition using an artificial neural network on a CPU/GPU cluster. \smallskip
%Used Principal Component Analysis for feature extraction and error backpropagation to train the network. %\medskip

%---------------------
%	PROJECT - 4
%---------------------
\textbf{Distributed Fault-Tolerant Stock Exchange System} \textit{using Java \& JGroups} \hfill April 2013
\begin{itemize}
    \item Implemented a stock exchange system that used fault-tolerant, virtually synchronous replicas to perform stock trades.
    \item Enhanced the system so that client information, trade requests \& stock data are preserved across node failures. %\medskip
\end{itemize}
%---------------------
%	PROJECT - 5
%---------------------
\textbf{Gossip based Topology Management in Peer-to-Peer Systems} \textit{using C \& Java} \hfill August 2013
\begin{itemize}
    \item Implemented the T-Man gossip based topology management protocol for peer to peer overlay networks, which achieved various topologies for a cluster of nodes based on different distance functions.
    \item Enhanced the T-Man implementation so that it would adapt it's topology at runtime to changes in the radius of the network. %\medskip
\end{itemize}
%---------------------
%	PROJECT - 6
%---------------------
%{\bf DNS Server} \textit{using Java RMI}\hfill{\bf February 2013} \smallskip \\
%Implemented a DNS server that supports recursive as well as iterative name resolution with support for server side caching using Java RMI. %\medskip

%---------------------
%	PROJECT - 7
%---------------------
%\textbf{Distributed Control of Currency Exchange Value} \textit{using Java} \hfill March 2013
%\begin{itemize}
%    \item Implemented totally-ordered multicasting using Lamport's logical clocks. 
%    \item Achieved consistent replicas across 3 machines, which shared a currency exchange value, using active replication protocol. %\medskip
%\end{itemize}
%---------------------
%	PROJECT - 8
%---------------------
%{\bf Simulation of Dijkstra's Algorithm} \hfill November 2010
%\begin{itemize}
%    \item Developed a program using C that would take a user-defined network graph as input and find the shortest path between any two nodes in the graph using %Dijkstra's algorithm. %\medskip
%\end{itemize}
%-----------------------------
%	PRESENTATION - 1
%-----------------------------
%{\bf ``Fundamentals of Switching \& Mobile Number Portability in a GSM network''}\hfill{\bf April 2011} \smallskip \\
%Prepared and presented a report on the operation of a large scale GSM network, including switching principles and mobile number portability %implementation. %\medskip

\end{document}
