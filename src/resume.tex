%---------------------------------------------
%			RESUME
%---------------------------------------------
\documentclass[10pt]{article}

%---------------------------------------------
%	BASIC PACKAGE DECLARATIONS
%---------------------------------------------
\usepackage[top=0.5in, bottom=0.5in, left=0.5in, right=0.5in]{geometry}
%\usepackage{array, xcolor}
%\usepackage{mathpazo}
%\usepackage[parfill]{parskip} % Removes paragraph indentation, but screws up the alignment of hfill. Use the next line
\usepackage{parskip}
%\usepackage{hyperref}
%\usepackage{fancyhdr}
\usepackage{enumitem}
\setlist{nolistsep} % No space between list items

%\addtolength{\oddsidemargin}{-.975in}
%\addtolength{\evensidemargin}{-.975in}
%\addtolength{\textwidth}{1.95in}
%\addtolength{\topmargin}{-1.1in}
%\addtolength{\textheight}{2.75in}

%\pagestyle{fancy}

%\pagenumbering{gobble} % Adding this to suppress page numbering while keeping footer

%\lfoot{All source code, including the \LaTeX\ source for this document can be found at \texttt{https://github.com/vivitsu}}
%\renewcommand{\headrulewidth}{0.4pt}
%\renewcommand{\footrulewidth}{0.4pt}

\begin{document}
%---------------------------------------------
%		     TITLE
%---------------------------------------------
\begin{center}
\textbf{\Large Vivitsu Maharaja}
\date{}
\thispagestyle{empty}
\smallskip \\
%---------------------------------------------
%	     CONTACT INFORMATION
%---------------------------------------------
3800 SW 34\textsuperscript{th} Street, Apt. Z251, Gainesville, FL $\diamond$ (352) 278-5449 $\diamond$ \texttt{vmaharaja@ufl.edu} $\diamond$ \texttt{https://github.com/vivitsu}
\end{center}

%---------------------------------------------
%	           OBJECTIVE
%---------------------------------------------
%\textbf{Objective}
%\smallskip
%\hrule
%To obtain a position where I can leverage and enhance my skills \& experience while working on challenging problems in computer engineering \& software development.

%---------------------------------------------
%	           EDUCATION
%---------------------------------------------
\textbf{Education}
\smallskip
\hrule
{\bf Master of Science}, Electrical \& Computer Engineering \hfill Expected graduation - May 2014\\
University of Florida, Gainesville, FL.\\
{\bf GPA:} 3.37/4.0

{\bf Bachelor of Engineering}, Electronics \& Communication \hfill May 2011\\
Dharmsinh Desai University, Nadiad, India.\\
{\bf GPA:} 62/100

Coursework: Computer Architecture, Parallel Computer Architecture, Computer Networks, Distributed Computing, Cloud Computing, Autonomic Computing, Virtual Computers.

%---------------------------------------------
%	           SKILLS
%---------------------------------------------
\textbf{Skills}
\smallskip
\hrule
\begin{itemize}
    \item C, Java, Go, MPI, CUDA, Hadoop, Riak, Redis, Solr.
    \item Knowledge of Data Structures \& Algorithms, Embedded Systems, Virtual Networks (VLANs).
\end{itemize} 
%Operating Systems: GNU/Linux, OS X, Windows

%---------------------------------------------
%		WORK EXPERIENCE
%---------------------------------------------
\textbf{Experience}
\smallskip
\hrule
\textit{Research Volunteer}, {\bf ACIS Lab}, University of Florida \hfill May 2013 - Present
\begin{itemize}
%---------------------
%	PROJECT - 1
%---------------------
    \item Set up and administered a \textbf{Solr}, \textbf{Hadoop} and \textbf{Riak} 
clusters as part of a project to benchmark information retrieval systems. As part of the Solr setup, created a schema to specify biological specimen records.
    \item Developed a Java application to store the specimen records in Solr and Riak, using the \texttt{\bf solrj} and \texttt{\bf riak-java-client} client libraries.
    \item Designed the application to fetch and parse \textbf{JSON} specimen records from a web service, convert the JSON to POJO/SolrDocument, upload it to the Riak/Solr
setup and benchmark each step of the process.
    \item Evaluated the Solr setup by benchmarking various parameters like indexing time, compression ratio, recall and the performance of the indexing algorithm.
%    \item Developing \& evaluating text-based information retrieval systems using frameworks like Hadoop, Riak \& Apache Solr.% \smallskip \\
\end{itemize}

\textit{Embedded Engineer}, {\bf Volansys Technologies}, Ahmedabad, India \hfill Nov. 2011 - July 2012
\begin{itemize}
%---------------------
%	PROJECT - 1
%---------------------
    \item Developed a \textbf{USB 2.0 (Enhanced Host Controller Interface)} Host Controller driver in \textbf{x86 assembly}, as part of an application which allowed clients to \textbf{PXE (Pre-boot eXecution Environment)} boot via a network using an USB to Ethenet adapter. 
    \item Enhanced the driver to manage the complete \textbf{state machine} of the controller including device detection, power management and data transfer. %\smallskip \\
%    \item Developed software components to manage the PCI bus, negotiate ownership with the BIOS subsystem and added support for multiple vendor adapters, as part of the overall PXE application. %\smallskip \\
%Handled responsibility for all aspects of the project, including development, maintenance, documentation \& client interaction.\smallskip \\
%---------------------
%	PROJECT - 2
%---------------------
    \item Developed a software feature for that would allow multiple broadcast domains in a wireless router to form a \textbf{VLAN (Virtual LAN)}.
\end{itemize}

%---------------------------------------------
%	          PROJECTS
%---------------------------------------------
\textbf{Projects}
\smallskip
\hrule
%---------------------
%	PROJECT - 1
%---------------------
\textbf{Distributed File System} \textit{using Java} \hfill August 2013 - December 2013
\begin{itemize}
    \item Designed and implemented a distributed, decentralized file system based on a peer-to-peer architecture.
    \item Implemented modular network management, file management and cluster management daemons to ensure consistency and fault-tolerance.
    \item Designed and implemented a multi-threaded client module and application that communicates with the file system and stores data on the cluster.
    \item Source code can be found at \texttt{https://github.com/vivitsu/Aether}.
\end{itemize}

%---------------------
%	PROJECT - 2
%---------------------
\textbf{Web Service for Location Based Applications} \textit{using Go \& Redis} \hfill April 2013
\begin{itemize}
    \item Designed and implemented a secure web application that allows a user to view different resources on the web about his location. User account details, location history \& a POI database were managed in Redis.
    \item Project source code can be found at \texttt{https://bitbucket.org/vivitsu/goserve}.
\end{itemize}

%---------------------
%	PROJECT - 3
%---------------------
%{\bf Face Recognition using Artificial Neural Networks} \textit{using C \& CUDA}\hfill{\bf February 2013 - April 2013} \smallskip \\
%Designed and implemented an application that performed face recognition using an artificial neural network on a CPU/GPU cluster. \smallskip
%Used Principal Component Analysis for feature extraction and error backpropagation to train the network. %\medskip

%---------------------
%	PROJECT - 4
%---------------------
\textbf{Distributed Fault-Tolerant Stock Exchange System} \textit{using Java \& JGroups} \hfill April 2013
\begin{itemize}
    \item Implemented a stock exchange system that used fault-tolerant, virtually synchronous replicas to perform stock trades.
    \item Enhanced the system so that client information, trade requests \& stock data are preserved across node failures. %\medskip
\end{itemize}

%---------------------
%	PROJECT - 5
%---------------------
%\textbf{Gossip based Topology Management in Peer-to-Peer Systems} \textit{using C} \hfill January 2013
%\begin{itemize}
%    \item Implemented the T-Man gossip based topology management protocol for peer to peer overlay networks, which achieved various topologies for a cluster of nodes based on different distance functions.
%    \item Enhanced the T-Man implementation so that it would adapt it's topology at runtime to changes in the radius of the network. %\medskip
%\end{itemize}

%---------------------
%	PROJECT - 6
%---------------------
%{\bf DNS Server} \textit{using Java RMI}\hfill{\bf February 2013} \smallskip \\
%Implemented a DNS server that supports recursive as well as iterative name resolution with support for server side caching using Java RMI. %\medskip

%---------------------
%	PROJECT - 7
%---------------------
%\textbf{Distributed Control of Currency Exchange Value} \textit{using Java} \hfill March 2013
%\begin{itemize}
%    \item Implemented totally-ordered multicasting using Lamport's logical clocks. 
%    \item Achieved consistent replicas across 3 machines, which shared a currency exchange value, using active replication protocol. %\medskip
%\end{itemize}

%---------------------
%	PROJECT - 8
%---------------------
%{\bf Simulation of Dijkstra's Algorithm} \hfill November 2010
%\begin{itemize}
%    \item Developed a program using C that would take a user-defined network graph as input and find the shortest path between any two nodes in the graph using %Dijkstra's algorithm. %\medskip
%\end{itemize}

%-----------------------------
%	PRESENTATION - 1
%-----------------------------
%{\bf ``Fundamentals of Switching \& Mobile Number Portability in a GSM network''}\hfill{\bf April 2011} \smallskip \\
%Prepared and presented a report on the operation of a large scale GSM network, including switching principles and mobile number portability %implementation. %\medskip

Other projects include \textbf{Gossip based Topology Management in Peer-to-Peer Systems} \textit{using C}, \textbf{Face Recognition using Artificial Neural Networks} \textit{using C \& CUDA}, 
\textbf{DNS Server} \textit{using Java RMI} \& \textbf{Totally-ordered Multicasting using Lamport logical clocks} \textit{using Java}. 

\end{document}
