%---------------------------------------------
%			RESUME
%---------------------------------------------
\documentclass[10pt, a4paper]{article}

%------------------------
%  PACKAGE DECLARATIONS
%------------------------
\usepackage[top=0.5in, bottom=0.5in, left=0.5in, right=0.5in]{geometry}
\usepackage{parskip}
\usepackage{enumitem}
\setlist{nolistsep} % No space between list items

% Add filled diamond instead of outlined diamond
\DeclareSymbolFont{extraup}{U}{zavm}{m}{n}
\DeclareMathSymbol{\vardiamond}{\mathalpha}{extraup}{87}

\begin{document}
%---------------------------------------------
%		     TITLE
%---------------------------------------------
\begin{center}
\textbf{\Large Vivitsu Maharaja}
\date{}
\thispagestyle{empty}
\smallskip \\
%---------------------------------------------
%	     CONTACT INFORMATION
%---------------------------------------------
$\vardiamond$ vivitsu@vivitsu.org $\vardiamond$ (352) 278-5449 $\vardiamond$ \texttt{https://github.com/vivitsu}
\end{center}

%---------------------------------------------
%	           EDUCATION
%---------------------------------------------
\textbf{EDUCATION}
\smallskip
\newline
{\bf Master of Science}, Electrical \& Computer Engineering \hfill May 2014\\
University of Florida, Gainesville, FL. \hfill GPA: 3.33/4.0

{\bf Bachelor of Engineering}, Electronics \& Communication \hfill May 2011\\
Dharmsinh Desai University, Nadiad, India. \hfill GPA: 62/100

%---------------------------------------------
%		WORK EXPERIENCE
%---------------------------------------------
\textbf{EXPERIENCE}
\smallskip
\newline
\textbf{Software Development Engineer}, Amazon Web Services, Seattle, WA \hfill Jul 2016 - Present
\begin{itemize}
\item \textbf{Transactional Services} \hfill Aug 2020 - Present
  \begin{itemize}[label=$\bullet$]
  \item Working on scaling the storage layer of an internal-only distributed, log-first database used by AWS teams to support higher read/write throughput and dataset sizes.
  \item Switched our database snapshot storage format from GZIP to LZ4 compression, leading to faster upload times during backup, and faster restore times during bootstrap.    
  \item Benchmarked effects of LZ4 compression on CPU utilization and network throughput in a variety of backup and restore scenarios.
  \item Benchmarked effects of partitioning data within our storage layout (using a consistent hashing scheme) on read/write throughput and database snapshot restore times.
  \end{itemize}
\item \textbf{SDKs and Tools} \hfill Jan 2018 - Jul 2020
  \begin{itemize}[label=$\bullet$]
  \item Worked on AWS' internal SDK release automation platform, which was responsible for orchestrating SDK release workflows, such as linting and validating AWS service models, release schedule management and compliance with AWS API guidelines. Apart from working on the orchestration service, I was also involved in the operational readiness and security review of the platform.
  \item Significantly improved the platform notification layer, to communicate upcoming launch deadlines, reviewer churn, and enforce that launch action items are complete, e.g. testing new SDK versions before release.
  \item Improved integration with SDK build systems by building a metadata specification API, using which SDK teams can customize their build systems, e.g. specify branches related to specific service builds in SDK codegen repos, service specific configuration for individual SDKs, e.g., PowerShell, etc.
  \item Idenitified bottlenecks in our service's day-to-day operations, and contributed to goals for reducing the operational load and oncall activities for the service.
  \item Led the security and operation readiness review for our service, which includes classifying all known architectural and operational security, latency and availability risks, and coming up with a plan to mitigate them.
  \end{itemize}
\item \textbf{Amazon Workdocs} \hfill Jul 2016 - Jan 2018
  \begin{itemize}[label=$\bullet$]
    \item Designed \& implemented a web publishing pipeline for the Amazon WorkDocs product blog using the Jekyll framework - https://blog.amazonworkdocs.com/.
  \end{itemize}
\end{itemize}
\textbf{Software Engineer}, LendingHome, San Francisco, CA \hfill Feb 2015 - Jul 2016
\begin{itemize}
\item Designed \& implemented a framework to schedule ETL (Extract, Transform, Load) pipelines. Using this framework, we were able to improve performance of our existing pipelines by more than 100\%.
\item Developed \& maintained a web service using Tesseract to automatically perform OCR on documents that are uploaded to the platform. The OCR-ed documents are then annotated by underwriters \& auditors to speed up loan processing.
\end{itemize}
\textbf{Software Engineer}, Applied Intelligence, IO Data Centers, San Francisco, CA \hfill Sep 2014 - Feb 2015
\begin{itemize}
\item Developed data processing pipelines using \textbf{Apache Pig} to analyze sensor data gathered from colocation centers.
\end{itemize}
\textbf{Embedded Engineer}, Volansys Technologies, Ahmedabad, India \hfill November 2011 - July 2012
\begin{itemize}
\item Developed \& maintained a \textbf{USB 2.0 (EHCI)} driver, to allow clients on a LAN to boot using an USB to Ethenet adapter, including interfacing with the \textbf{PCI} \& \textbf{BIOS} subsystems in order to manage the host controller \textbf{state machine} \& maintain driver compatibility with adapters from multiple vendors. 
\end{itemize}

%---------------------------------------------
%	          PROJECTS
%---------------------------------------------
\textbf{PROJECTS}
\smallskip 
\newline
% ------- STATIC SITE GENERATOR PROJECT ------
%% \textbf{Static Site Generator} using Java \hfill May 2020 - Ongoing
%% \begin{itemize}
%% \item Currently working on a static site generator in my spare time. The goal for this project is to write a complete static site generator, including markdown and template parsing and processing, without using any external dependencies.
%% \end{itemize}
% ------- STATIC SITE GENERATOR PROJECT ------
\textbf{URL Shortener} using C, JavaScript \hfill November 2019
\begin{itemize}
\item Implemented a URL shortener in C in my spare time. The goal for this project to improve my C and web development skills, while using minimal external dependencies or libraries.
\end{itemize}

%---------------------------------------------
%	           SKILLS
%---------------------------------------------
\textbf{SKILLS}
\smallskip 
\begin{itemize}
\item Programming Languages: Comfortable in Java and Kotlin. Familiar with C and Python with a little experience in Ruby and JavaScript
\end{itemize} 

%------------------
%    COURSEWORK
%------------------
%% \textbf{COURSEWORK}
%% \smallskip 
%% \newline
%% Computer Architecture, Probability \& Random Processes, Computer Networks, Distributed Computing, Cloud Computing, Autonomic Computing, Information Retrieval (Guided Research), Virtual Computers.
\end{document}
